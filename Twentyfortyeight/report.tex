\documentclass[a4paper, 12pt]{article}
%\usepackage[a4paper,hmargin=3cm,vmargin=3cm]{geometry}
\usepackage[english]{babel}
\usepackage[applemac]{inputenc}
\usepackage[sc]{mathpazo}
\linespread{1.05}         % Palatino needs more leading (space between lines)
\usepackage[T1]{fontenc}

\usepackage{amsmath}
\usepackage{hyperref}

\newcommand{\link}[1]{\href{http://#1}{\texttt{#1}}}

\newcommand{\code}[1]{\texttt{#1}}

\title{Reinventing 2048 \\ \large{An ITSMAP Project Report}}
\author{Kristoffer Andersen \\ \large{20103316} \and
Christian Clausen \\ \large{20081015}}
\date{$6^{th}$ of June, 2014}

\begin{document}
% Front page
\pagenumbering{gobble}
\maketitle
\newpage
% TOC
\tableofcontents
\newpage
\pagenumbering{arabic}

\section{Introduction}
The game known as 2048 went viral in 2014 when Gabrielle Cirulli
posted a simple game with a 4 by 4 grid with the squares from 2 to
2048 on Github. The gameplay was self-explanatory, the visuals
timeless and a viral hit was born.

Since then the game has spawned numerous variations, but for a game
with such obviously touch-friendly input, the variations in the mobile
market has been few and far between.

This report presents a reimplementation of Cirulli's 2048 from scratch
in the native Android Framework, but with a twist: we wholly embrace
the prospects that the app is running on a mobile device.

We proceed with detailing a specification of the problem, and
explicating in what ways our vision is different from the original. We
then go on to sketch some of the architectural decisions that we made
in order to facilitate our design. Finally, we detail the Android
Framework features that we made use of, in addition to the features
that we \emph{could} have made use of. Additionally, we bring a work
plan in the appendix, detailing who did what on the project.

\section{Specification}
% Requirements specification (Comments and sketches like GUI)
The goal with our app is to provide the basic 2048 experience. As
such, we here briefly outline the classic game mechanics, for the
readers that are not familiar with it.

\subsection{Core Game Mechanics}

The game is played on a quadratic grid with four squares to a
side. During the course of the game, the grid is filled with tiles
that can be slid across the grid. The only four actions the player can
take is to slide all tiles across the grid in one of the cardinal
directions: up, down, left and right. \emph{Playing} the game thus
amounts to deciding a sequence of these directions.

The tiles are numbered with subsequent powers of 2, starting with
2. When two tiles with the same value collide, they merge to a new
tile with the sum of the two involved numbers. That is, as tile with
the number 32 collides with another tile with the number 32, they
merge to form a new tile with the number 64.

The objective of the original game is to form a tile with the number 2048. When
two tiles merge, the player is also awarded the sum of the two tiles
in points.

When the player has made a decision, resulting in a changed board, the game places a new tile in a randomly
chosen empty cell, with the value 2 or 4.

The dynamics of the game thus arise as the board fills up, making it
more difficult for the player to merge tiles. And, as the game
progresses, the average value of the tiles already on the board
increases, and as such it becomes more difficult to use the new 2 and
4 tiles the game places.

The game is over when there are no more moves that would result in a
tile moving or two tiles merging.

\subsection{Variations}

The game as described in the previous section is how Cirulli published
it in 2014, with a 4 by 4 grid and the tiles from 2 to 2048. It was
implemented in JavaScript, and used the arrow keys as input. Simple
\emph{WebView}-style apps were made for the Android and iOS platforms
with touchbased input, where a swipe in the cardinal directions
results in the corresponding move in the game.

However, the fact that Cirulli decided to publically release the
intellectual proprties associated with the game, the fact that the
game is hosted on GitHub (such that with one click, one gets a
complete, hosted copy of Cirulli's game to alter as one sees fit, live
and on the internet from the very first moment), combined with its
accessibility, and addictiveness has meant that it has inspired
countless variations.

Some variations extend the dimensionality of the grid on which the
game is played. If the game as envisioned by Cirulli is played on a
plane, the 3D version is played \emph{in} a cube, and so on. We lack
prepositions in the english language to describe how games are played
in four dimensions and beyond, but the variations exist, readily up to
at least 6 dimensions. To be able to fathom and play such a game is
one thing; from a technical stand-point, it is interesting to consider
how to control such a game. While the arrow keys, or a swipe on a
touchscreen are completely believeable analogies to movement in two
dimensions, it becomes much more difficult to come up with controls by
analogy once the game extends three dimensions. Most implementations
simply give the player yet another pair of keys to control movement
alog the given dimensions. We will return to this point shortly.

Other variations play on the aesthetics of the app, with pictures,
popular internet memes, operators from the creator's favorite
programming language, and so on. While entertaining and clearly a
contributing part of the game as a phenomenon, it is from a technical
perspective not very interesting.

Some varitions are outside of a general description. A favourite
implementation of ours is a democratized version: every player dials
in to the server and ``votes'' for what the next move will be, as
opposed to playing it directly. The social angle is defnitely a
contributing factor to the success of the app, and this is playing
that to the extreme.

It would not surprise us if the number of variants is approaching
2048; there are far too many to cover them here. For a more complete
list we refer to \link{http://phenomist.wordpress.com/2048-variants/}.

\subsection{Vision}

We wish to investigate the input design-space. The idea originated
from brainstorming what would make it possible to exhibit the features
of the Android Framework and a mobile platform in a fun and obvious
way.

To make a general statement about the controls required for the game,
we drew inspiration from the higher dimensionality variations that
have been made, where they rather weakly just added another keypair to
the set of controls. As such it becomes obvious that an input device
for a game 2048 must just be able to produce two discernible, discrete
signals in order to control movement along an axis. 

Our vision with this prototype is to build an app that allows the
prototyping of input control methods to a game of 2048. Mobile devices
offer such a wealth of sensory input that a classic desktop setting
simply cannot compete. 

\subsection{Specification}

The app has to...
\begin{itemize}
\item ... provide the basic functionality, look and feel of a game of 2048.
\item ... allow the user to select different input methods.
\item ... provide an on-line highscore service, in order to make use
  of the connectivity features of the Android Platform and to enchance
  the social aspect of the game.
\item ... provide appliaction developers with a framework for
  prototyping new input method variations for games of 2048.
\end{itemize}

\section{Software Architecture}

So there are two intrinsic forces guiding our hand in the design of
this app. For one, it is an important consideration that we are
working on the Android Platform. It has unavoidable consequences while
offering unique opportunities. Secondly, we wish to build as flexible
an app as possible, in order to facilitate the 

The basic game logic is handled through a software architecture
classic: the Model View Control pattern. The pattern is a OO oldie,
and an architecture centered around the MVC pattern gives a range of
benefits. Primarily it solves two problems given one assumption: data
is shared. Problems then arise when we realize that the ways we wish
\emph{to manipulate} and \emph{to visualize} it is variable, perhaps
even undetermined. The MVC pattern then defines three \emph{roles}:
the model, the view and the controller. The model has the
responsibilities of maintaining the state of the data, and to notify
the data visualizers (the views) when the data changes. The views have
the responsibility of visualizing data, and possibly pass on the user
interactions with the views as events to the controllers, who bear the
responsibility of translating these events into state changes on the
model.

% Assumptions for and explanatins of chosen design
% UML + Sequence diagrams of key components in the APP

\section{Uses of the Android Framework}
% We have used
% - Activities + Fragments
% - Preferences
% - Externalization
% - Background tasking / Async
\subsection{Background Tasks}
In parts of the project we wish to display information, that is not
readily avalible, such as the license files, stored on the phone, or
the highscore stored on a server. In these cases it makes sence to
collect and prepare the information in a background task, so the user
isn't nessasarily bothered by any possible wait there might be while
loading the data.

% - External services (RESTful servers) (+ Content Provider?)
\subsection{External Services}
In the development of the project we have used multiple external
services, each treated hereunder.

\subsubsection{RESTful Server}
The highscore information stored on the server can be accessed through
the JSON format, which we have previously worked with in the
course. Thus we here follow the same structure: Create a URL, then a
GetRequest, convert the input stream to a string, and then use Java's
built-in libraries for JSON to parse the string. This way we could
easily extend the project to store other informations such as the
geolocation of each highscore, names, time of day, and so on. We could
even extend the server with any of these, without changing anything in
the Android code.

% chr: Files is an "external service"? Is that a "Content Provider"?

% - intents
% - Managers needed for device sensors
% - Toasts
\subsection{Toast}
A Toast is part of the Android framework, is a temporary pop-up
message. We use Toast messages to notify the user that his highscore
has been sent to the server.

% - Custom views
% We could have used
% - Network connections
% - Local storage?
% - Notifications

\section{Conclusion}
% Conclusion

\newpage
\appendix
\section{Appendix: Workplan}
\subsection{Christian's Contribution}
\begin{itemize}
\item \code{src/dk/compsci/twentyfortyeight/}
  \begin{itemize}
  \item \code{HighScoreActivity.java}
  \item \code{EngineImpl.java}
  \end{itemize}
\item \code{res/layout/}
  \begin{itemize}
  \item \code{activity\_highscore.xml}
  \end{itemize}
\item The REST service running at \code{http://2048.compsci.dk/json.php}, and the use hereof.
\end{itemize}
\subsection{Kristoffer's Contribution}
\begin{itemize}
\item Everything not mentioned under \textbf{Christian's Contribution}
\end{itemize}

\end{document}