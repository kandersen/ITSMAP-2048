\documentclass[a4paper, 12pt]{article}
%\usepackage[a4paper,hmargin=3cm,vmargin=3cm]{geometry}
\usepackage[english]{babel}
\usepackage[applemac]{inputenc}
\usepackage[sc]{mathpazo}
\linespread{1.05}         % Palatino needs more leading (space between lines)
\usepackage[T1]{fontenc}

\usepackage{amsmath}

\newcommand{\code}[1]{\texttt{#1}}

\title{Reinventing 2048\\\large{An ITSMAP Project Report}}
\author{Kristoffer Andersen\\\large{20103316} \and
Christian Clausen \\ \large{20081015}}
\date{$6^{th}$ of June, 2014}

\begin{document}
\maketitle
\nonumber
\newpage
\tableofcontents
\newpage
% Front page
% TOC
\section{Introduction}
The game known as 2048 went viral in 2014 when Gabrielle Cirulli
posted a simple game with a 4 by 4 grid with the squares from 2 to
2048 on Github. The gameplay was self-explanatory, the visuals
timeless and a viral hit was born.

Since then the game has spawned numerous variations, but for a game
with such obviously touch-friendly input, the variations in the mobile
market has been few and far between.

This report presents a reimplementation of Cirulli's 2048 from scratch
in the native Android Framework, but with a twist: we wholly embrace
the prospects that the app is running on a mobile device.

We proceed with detailing a specification of the problem, and
explicating in what ways our vision is different from the original. We
then go on to sketch some of the architectural decisions that we made
in order to facilitate our design. Finally, we detail the Android
Framework features that we made use of, in addition to the features
that we \emph{could} have made use of. Additionally, we bring a work
plan in the appendix, detailing who did what on the project.

\section{Specification}
% Requirements specification (Comments and sketches like GUI)
The goal with our app is to provide the basic 2048 experience. As
such, we here briefly outline the classic game mechanics, for the
readers that are not familiar with it.

\subsection{Core Game Mechanics}

The game is played on a quadratic grid with 4 squares to a
side. During the course of the game, the grid is filled with tiles
that can be slid across the grid. The 4 only actions the player can
take is to slide all tiles across the grid in one of the 4 cardinal
directions: up, down, left and right. \emph{Playing} the game thus
amounts to deciding a sequence of these directions.

The tiles are numbered with subsequent powers of 2, starting with
2. When two tiles with the same value collide, they merge to a new
tile with the sum of the two involved numbers. That is, as tile with
the number 32 collides with another tile with the number 32, they
merge to form a new tile with the number 64.

The objective is the game is to form a tile with the number 2048. When
two tiles merge, the player is also awarded the sum of the two tiles
in points.

When the player has made a decision, slid the tiles and whatever
merges may occur has occured, the game places a new tile in a randomly
chosen empty cell, with the value 2 or 4.

The dynamics of the game thus arise as the board fills up, making it
more difficult for the player to merge tiles. And, as the game
progresses, the average value of the tiles already on the board
increases, and as such it becomes more difficult to use the new 2 and
4 tiles the game places.

The game is over when there are no more moves that would result in a
tile moving or two tiles merging.

\subsection{Variations}

The game as described in the previous section is how Cirulli published
it in 2014, with a 4 by 4 grid and the tiles from 2 to 2048. It was
implemented in JavaScript, and used the arrow keys as input. Simple
\emph{WebView}-style apps were made for the Android and iOS platforms
with touchbased input, where a swipe in the cardinal directions
results in the corresponding move in the game.

However, the fact that Cirulli decided to publically release the
intellectual proprties associated with the game, the fact that the
game is hoste don GitHub (such that with one click, one gets a
complete, hosted copy of Cirulli's game to alter as one sees fit, live
and on the internet from the very first moment), combined with its
accessibility has meant that it has inspired countless variations.

Some variations extend the dimensionality of the grid on which the
game is played.



\section{Software Architecture}
% Assumptions for and explanatins of chosen design
% UML + Sequence diagrams of key components in the APP

\section{Uses of the Android Framework}
% We have used
% - Activities + Fragments
% - Preferences
% - Externalization
% - Background tasking / Async
% - External services (RESTful servers) + Content Provider
% - intents
% - Managers needed for device sensors
% - Toasts
% - Custom views
% We could have used
% - Network connections
% - Local storage?
% - Notifications

\section{Conclusion}
% Conclusion

\newpage
\section{Appendix: Workplan}
\subsection{Christian's Contribution}
\begin{itemize}
\item \code{src/dk/compsci/twentyfortyeight/}
  \begin{itemize}
  \item \code{HighScoreActivity.java}
  \item \code{EngineImpl.java}
  \end{itemize}
\item \code{res/layout/}
  \begin{itemize}
  \item \code{activity\_highscore.xml}
  \end{itemize}
\item The REST service running at \code{http://2048.compsci.dk/json.php}
\end{itemize}
\subsection{Kristoffer's Contribution}
\begin{itemize}
\item The report
\item Everything not mentioned under \textbf{Christian's Contribution}
\end{itemize}

\end{document}